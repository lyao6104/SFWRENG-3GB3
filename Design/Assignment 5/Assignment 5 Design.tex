\documentclass[10pt]{article}
\usepackage[utf8]{inputenc}
\usepackage[default]{lato}
\usepackage[T1]{fontenc}
\usepackage[margin=1in]{geometry}
\usepackage{listings}
\usepackage{graphicx}
\usepackage{hyperref}
\usepackage{xcolor}
\usepackage{amsmath}
\usepackage{amsthm}
\usepackage{amssymb}

\setlength{\parindent}{0em}
\setlength{\parskip}{1em}

\definecolor{linkcolour}{RGB}{0, 149, 255}
\hypersetup{
    colorlinks=true,
    linkcolor=linkcolour
}

\begin{document}

\title{\textit{Awakening}}
\author{Luna Y.}
\date{
    SFWRENG 3GB3 -- Assignment 5\\[0.25cm]
    \today
}
\maketitle

\tableofcontents

\section{Basic Information}

As specified in the assignment outline, this game will be built off of Unity's RPG Creator Kit.
The game itself will be one where the player must gather materials to craft items and use them to defeat enemies.
Different enemies will have different types, and different types of items (weapons, armour, etc.) will be more or
less effective against different types of enemies (somewhat similar to type advantages from the \textit{Pokemon} games).

As for challenges, there will resource accumulation challenges, stealth challenges, as well as survival and reduction of enemy forces.
There are also elements of spatial awareness challenges. The specific details of these challenges will be explained in the
later sections of this document.

\section{Gameplay}

In the level, there are enemies scattered across the map that have predetermined movement patterns. Enemies, as well
as the player, each have their own health values, which can be decreased by being attacked. Players kill enemies
by reducing their health to zero, and players die when \textit{their} health reaches zero. The goal is to defeat
all the enemies, while keeping the player avatar alive. If enemies have set movement patterns,
this also adds an element of a pattern recognition challenge.

However, combat is not just as simple as running up to an enemy and hitting it until it dies. Each enemy has a particular
``type'': armoured, magical, unarmoured, and incorporeal, and the player's weapons can either be blunt, piercing, or magical.
Strengths and weaknesses are given in the following table, where ``Weak'' indicates an armour type's weakness to an attack,
``Resist'' indicates that an armour type is highly resistant against an attack, ``Invulnerable'' indicates that attacks of
that type will do zero damage to the corresponding armour type, and ``Normal'' indicates no modifier:
\begin{center}
    \begin{tabular}{|c|c|c|c|c|}
        \hline
        & Armoured & Magical & Unarmoured & Incorporeal\\
        \hline
        Blunt & Resist & Normal & Normal & Invulnerable\\
        Piercing & Weak & Normal & Normal & Invulnerable\\
        Magical & Normal & Resist & Normal & Weak\\
        \hline
    \end{tabular}
\end{center}
The player starts out with no weapon equipped, so they must first gather resources to craft one. Players store all
crafted weapons in their inventory, and different weapons can be equipped or unequipped at will.
While the player has no weapon, they are unable to attack enemies, which means they must remain in stealth,
and avoid combat with enemies to gather resources for their first weapon.

While crafting, the player needs to be able to keep track of their surroundings, and also quickly stop crafting
and react to any changes if necessary. As such, the crafting itself will be rather simple, and should be interruptible
as well. Crafting different weapons will require different amounts of various types of materials, which must be gathered
by the player during gameplay. Gathering materials is as simple as moving next to a material node on the map and interacting
with it to gather the material. Both crafting and gathering items should require a certain amount of time,
which encourages the player to not perform either action while there are enemies nearby that might interrupt them.
Just as an example, crafting in this game will be somewhat similar to how crafting works in \textit{Starbound};
each crafting recipe requires a list of materials, and different items will take different amounts of time to craft.
This directly contributes to the resource accumulation and spatial awareness challenges. Resource accumulation is fairly
straightforward, and spatial awareness comes from the player needing to track where they have been, where things are
in the map, what enemies are nearby, and so on.

For simplicity, all combat will be performed using ranged attacks. Each weapon has a damage and type (see table above)
associated with it, and the exact damage done to an enemy is further modified by any damage bonuses or penalties
(which are determined by type matchings). For simplicity, the player is not weak or resistant to any particular type of damage.
Additionally, rather than free targeting of attacks, the mechanic of ``tab targeting'', commonly seen in MMORPGs like \textit{Final Fantasy XIV},
will be borrowed, so that players will select an explicit target for their attacks (however, the method of selecting targets doesn't
necessarily have to be the Tab key, and mouse clicks can also be used to target enemies). While an enemy is targeted,
the player should also be able to see a visualization of that enemy's aggression radius (outside of which the enemy will not attack the player),
which should help them plan their routes for gathering materials, and finding safe places to craft.

\section{Aesthetics}

The story/lore of the game is that the player is some kind of adventurer/hero who has awoken in an underground setting (either a cave system
or castle) filled with enemies, and they must escape after defeating all the enemies near them. This lore will be presented
either as an introduction pop-up or in the instructions.

The visual and musical styles should match the story. That is, the visuals should be darker and more ``gloomy'',
while the music used should be more instrumental, with a darker/more sombre feeling, as opposed to bright and cheerful.

\pagebreak

\section{Player Experience}

The initial section of gameplay should evoke a sense of suspense and ``sneakiness'', as the player must navigate the map
and gather resources without being spotted by enemies. The player should feel rather weak at this point, as they have no
method of fighting back against enemies. Since the player will likely feel frustrated if they make a mistake and are permanently
followed by an enemy that spotted them, the player should be able to escape enemies by running some distance away (this distance
should be greater than the distance at which enemies notice the player).

During combat, the mood should be more energetic, which can be supported by a change in music (just like in \textit{Gloamwardens}).
The player should have access to basic information about enemies (e.g. health, armour type, attack strength) to allow
for informed decision-making and a more strategic experience (for example, the player could plan to clear out an area
of relatively weaker enemies in order to gain access to new resources, while also avoiding areas with stronger enemies).

The player should feel a sense of progression throughout the level as they manage to create more powerful weapons.
At the beginning, they should feel relatively weak, while by the end, they should feel powerful.

% \section{Changes from Original}



\section{Works Used}

This project uses Unity's 2D RPG Creator Kit as a base. It can be found in the \href{https://assetstore.unity.com/packages/templates/tutorials/creator-kit-rpg-149309}{Unity Asset Store}.

% \subsection{Concepts}

% \subsection{Code}

% Like Assignment 2, the game features many script assets originally made for \textit{Nightscape},
% which is still not released publicly. This includes camera functionality, object selection, as well as
% random name generation. Many of the names themselves include segments generated from
% \href{https://www.fantasynamegenerators.com/}{Fantasy Name Generators}.
% The appropriate behaviours were also modified to use Unity's new input system, using concepts learned
% from \href{https://www.youtube.com/watch?v=PsAbHoB85hM}{this tutorial} as well as the Unity documentation.

% This game also builds on \textit{The Starlight Rebellion}'s version of the audio player originally made
% for \textit{Lightrunner}.

% The game also uses the \href{https://arongranberg.com/astar/}{A* Pathfinding Project} as its pathfinding algorithm,
% which uses the \href{https://unity3d.com/legal/as_terms}{Standard Unity Asset Store EULA}. Movement, attacking,
% and some other AI-related behaviour was heavily adapted from another one of my old personal projects called \textit{War of the Vale},
% which is also not publicly available.

% \subsection{Art}

% \textbf{\textit{Dungeon Crawl Stone Soup}}
% \begin{itemize}
%     \item The game uses main and supplementary tilesets from \textit{Dungeon Crawl Stone Soup}, hosted \href{https://opengameart.org/content/dungeon-crawl-32x32-tiles}{here}
%         and \href{https://opengameart.org/content/dungeon-crawl-32x32-tiles-supplemental}{here}, respectively.
%     \item These assets are in the public domain, under the \href{https://creativecommons.org/publicdomain/zero/1.0/}{CC0 1.0} license.
% \end{itemize}

% \subsection{Fonts}

% \textbf{Alegreya SC}
% \begin{itemize}
%     \item Licensed under OFL
%     \item Sourced from \href{https://fonts.google.com/specimen/Alegreya+SC}{Google Fonts}
% \end{itemize}

% \textbf{IM Fell Double Pica SC}
% \begin{itemize}
%     \item Licensed under OFL
%     \item Sourced from \href{https://fonts.google.com/specimen/IM+Fell+Double+Pica+SC}{Google Fonts}
% \end{itemize}

% \subsection{Music}

% Auroras of Saturn by MusicLFiles\\
% Link: \url{https://filmmusic.io/song/6782-auroras-of-saturn}\\
% License: \url{https://filmmusic.io/standard-license}

% Dawn by Alexander Nakarada\\
% Link: \url{https://filmmusic.io/song/4867-dawn}\\
% License: \url{https://filmmusic.io/standard-license}

% Horizon Flare by Alexander Nakarada\\
% Link: \url{https://filmmusic.io/song/4837-horizon-flare}\\
% License: \url{https://filmmusic.io/standard-license}

% Dungeon Crawl Stone Soup assets, Szadi RPG Cave and Rogue Castle.

\end{document}