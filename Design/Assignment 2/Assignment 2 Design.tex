\documentclass[10pt]{article}
\usepackage[utf8]{inputenc}
\usepackage[default]{lato}
\usepackage[T1]{fontenc}
\usepackage[margin=1in]{geometry}
\usepackage{listings}
\usepackage{graphicx}
\usepackage{hyperref}
\usepackage{xcolor}
\usepackage{amsmath}
\usepackage{amsthm}
\usepackage{amssymb}

\setlength{\parindent}{0em}
\setlength{\parskip}{1em}

\definecolor{linkcolour}{RGB}{0, 149, 255}
\hypersetup{
    colorlinks=true,
    linkcolor=linkcolour
}

\begin{document}

\title{\textit{The Starlight Rebellion}}
\author{Luna Y.}
\date{
    SFWRENG 3GB3 -- Assignment 2\\[0.25cm]
    \today
}
\maketitle

\tableofcontents

\section{Basic Information}

In this game, the player will play as the leader of a rebellion against an interstellar empire. The game itself will be a top-down style
turn-based strategy game with the gameplay dynamic of ``board control'' at its core. The tiles of the board will be the star systems
making up the aforementioned empire, and the game pieces are represented by fleets of starships. The player starts with a single fleet
in a single star system, and must use it to capture more systems and eventually win the game by controlling the entire board.

The game takes aesthetic inspiration from similar games such as \textit{Stellaris, Starcraft,} and \textit{Star Wars: Empire at War}.
The layout of the map will also be similar to that of the galaxy map of \textit{Empire at War}, but differs in that, where \textit{Empire at War}
has static maps that do not change, this game will have a single map whose layout will be randomly generated between different playthroughs,
possibly having some basic settings that affect difficulty as well. This will allow for more variety between playthroughs, as well as more replayability.

The combat of the game will be somewhat similar to \textit{Chess}. Each fleet can move to any adjacent system.
For simplicity, if a player moves a fleet onto a system containing a fleet from the opposing player, the attacking
fleet defeats the defending one and conquers the system. Any attacking fleet will conquer a system if they arrive unopposed as well.
However, the combat differs from \textit{Chess} in that there is only one type of unit, with basic movement mechanics (so nothing
similar to knights, bishops, etc.). This style of combat also encourages more strategic gameplay. For example, if an imperial fleet
is defending a system, and the adjacent systems were empty, the player will not be able to approach the defended planet without taking losses,
since moving in will leave the player's fleets vulnerable to an attack.

\section{Gameplay}

As mentioned above, the game board will consist of various tiles, represented by star systems, and pieces, represented by
fleets of starships. The star systems will be connected to one another via ``hyperlanes'', creating a structure that is essentially an undirected graph.
The tiles of the board should form a connected graph, but two systems might not necessarily be directly connected to one another.
The player begins the game with one fleet and one system, and must use their initial fleet to capture
more tiles in order to win the game. The AI opponent will start with considerably more resources at its disposal than the player,
which the player will have to overcome. The player may obtain more fleets by capturing/liberating systems from the AI,
while the AI will receive new fleets every couple of turns. This also fits the game's narrative rather well; the player is leading
a rebellion that has popular support, so liberating systems allows more people to join them, while the AI-controlled empire must
assemble new fleets the regular way.

The board will also be procedurally generated, in order to save time on level design and add variety between playthroughs.
Parameters such as the number of systems or number of hyperlanes should also be modifiable, which will also have an effect
on how difficult the game is.

The game will feature turn-based movement and combat, with each fleet being able to move independently from one another
(i.e. the player can move all their fleets in one turn, rather than only one fleet per turn). The combat will also be very simple,
with attacking fleets always winning over any defenders, as described above. Because of this, each tile of the board will also be limited
to holding one fleet at a time. This might cause issues with how new fleets are spawned, so in the event that a new fleet is trying to spawn
on an occupied system, it will instead be placed at the nearest unoccupied friendly system.

\section{Aesthetics}

The visuals of the game should look futuristic, and themed around science-fiction and space.
The user interface of the game should somewhat resemble a kind of spaceship/computer console
(similar to the Terran interface in \textit{Starcraft}).

The graphics of the game will be predominantly created using 2D sprites. That being said,
graphics are not extremely important for this game, and the visuals should be kept relatively simple
to save time.

The music should also have background music that is not too obtrusive, and this music should have
a futuristic, possibly militaristic feel to it.

The player will receive a brief text-based introduction to the game universe when they start a new game.
Story is not very important for this game, but the player does need an introduction to the game's setting,
however brief it may be.

\section{Player Experience}

Overall, the game should feel slower-paced and methodical. This is reinforced by the turn-based nature of the game,
which allows the player to take their time and really think strategically. As a side effect of this, it should
theoretically be less likely for the player to get frustrated at the game for reasons like being caught off-guard
by the AI, or losing a system or fleet, since they would presumably be able to predict these situations to a reasonable degree,
and course correct where necessary.

The player should also feel like they are overcoming some very difficult (but not necessarily insurmountable) odds in order to
achieve victory against the Empire. This is supported by the game's narrative, as well as the player's starting position.
Realistically, any uprising against a large interstellar state would be hard-pressed to succeed, and this is reflected by
the vast difference between the player, who starts with one system and a single fleet, and the AI, which starts with the rest of
the star systems and several more fleets. It is up to the player to overcome these odds and win, and if they do, they should
feel a sense of triumph as well. That being said, however, this difficulty will likely be indirectly adjustable through various
map generation settings before the game starts, for players who prefer a slightly more relaxed experience.

% \section{Changes from Original}

% The biggest change from the original design is that the average length of a single run was decreased from five minutes to about one to two minutes.
% The reason for this is that designing the tilesets to produce such long individual run times led to the gameplay being rather dull.
% So now, the tilesets are designed so that an average run lasts from about one to two minutes, and there are enough tilesets that a player
% should experience a decent variety in two or three runs. This means there is still about five minutes of content on average.

% As for aesthetics, the only changes were that jumping no longer has a sound effect, and there is no story/lore either.
% This is because they were viewed as unnecessary, and did not really fit in with the rest of the game either.

% This design document was also updated to reflect the given feedback, as well as to document any third-party assets and other such things that were used.

\section{Works Used}

\subsection{Concepts}

The map generation algorithm uses a technique involving grids in order to efficiently test for overlap, which was learned from
\href{https://www.youtube.com/watch?v=7WcmyxyFO7o}{this video}.

\subsection{Code}

The name generation script and object selection script was taken from a personal project of mine called \textit{Nightscape},
which is currently not released publicly. Many of the names themselves include segments generated from
\href{https://www.fantasynamegenerators.com/}{Fantasy Name Generators}.

This game also uses a lot of miscellaneous code originally created for \textit{Lightrunner}, including various shaders,
as well as the audio player.

The game also uses the \href{https://arongranberg.com/astar/}{A* Pathfinding Project} as its pathfinding algorithm,
which uses the \href{https://unity3d.com/legal/as_terms}{Standard Unity Asset Store EULA}.

\subsection{Art}

\textbf{UI Minimalism SciFi}
\begin{itemize}
    \item The game uses UI assets from the \href{https://opengameart.org/content/assets-ui-minimalism-scifi}{``UI Minimalism SciFi''} pack,
        created by \href{https://wenrexa.itch.io/}{Wenrexa}.
    \item These assets are in the public domain, under the \href{https://creativecommons.org/publicdomain/zero/1.0/}{CC0 1.0} license.
\end{itemize}

\subsection{Fonts}

\textbf{Nova Square}
\begin{itemize}
    \item Licensed under OFL
    \item Sourced from \href{https://fonts.google.com/specimen/Nova+Square}{Google Fonts}
\end{itemize}

\textbf{Spartan}
\begin{itemize}
    \item Licensed under OFL
    \item Sourced from \href{https://fonts.google.com/specimen/Spartan}{Google Fonts}
\end{itemize}

% \subsection{Music}

% "Furious Freak", "Super Power Cool Dude", "Voxel Revolution"\\
% Kevin MacLeod (\href{https://incompetech.com}{incompetech.com})\\
% Licensed under Creative Commons: By Attribution 3.\\
% \url{http://creativecommons.org/licenses/by/3.0/}

\end{document}