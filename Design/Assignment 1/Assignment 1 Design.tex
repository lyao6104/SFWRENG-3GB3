\documentclass[10pt]{article}
\usepackage[utf8]{inputenc}
\usepackage[default]{lato}
\usepackage[T1]{fontenc}
\usepackage[margin=1in]{geometry}
\usepackage{listings}
\usepackage{graphicx}
\usepackage{hyperref}
\usepackage{xcolor}
\usepackage{amsmath}
\usepackage{amsthm}
\usepackage{amssymb}

\setlength{\parindent}{0em}
\setlength{\parskip}{1em}

\definecolor{linkcolour}{RGB}{0, 149, 255}
\hypersetup{
    colorlinks=true,
    linkcolor=linkcolour
}

\begin{document}

\title{\textit{Lightrunner}}
\author{Luna Y.}
\date{
    SFWRENG 3GB3 -- Assignment 1\\[0.25cm]
    \today
}
\maketitle

\tableofcontents

\section{Basic Information}

In this game, the player will play as an amorphous entity composed of light. The player character is constantly travelling forwards,
and the only action available to the player is jumping. The player character is constantly shrinking due to expending their light,
and must collect more by moving close enough in order to survive and continue playing. However, collecting too much light will
cause the player to become unstable, and they will eventually dissipate once a certain point is reached, causing a game over as well.

The goal of the game is to survive as long as possible, which will require the player to collect enough light to survive,
while taking care to avoid obstacles and not collecting too much light. Each playthrough, from the start of a game to the inevitable game over,
will be referred to as a ``run''. The game concept takes inspiration from the ``endless runner'' genre of games,
which includes titles such as \textit{Geometry Dash, Temple Run, Jetpack Joyride}, and so on. As such, there is no explicit win condition,
but there will be two loss conditions; the player loses when they either fail to avoid an obstacle, or when they run out of light.
Specifically, this game will be borrowing quite a few aesthetic elements from \textit{Geometry Dash}, such as the simple geometric graphics,
and neon, futuristic visual style. From both \textit{Temple Run} and \textit{Jetpack Joyride}, the game borrows the concept of having
collectible items (e.g. coins) during gameplay, though it differs, as in those games, the collectible coins are mainly a currency
used to purchase upgrades, while in \textit{Lightrunner}, collecting light is directly tied to the players health, and cannot be used to purchase
anything.

In terms of level design, the game world will resemble a standard 2D platformer, except the player is constantly moving forwards (i.e. to the right),
and the only way to travel vertically is by jumping. The tiles of the level, as well as the light being collected by the player,
will be procedurally generated according to a collection of premade tilesets; this should allow for more variety while also keeping
the level structured and less chaotic.

\section{Gameplay}

The core mechanic of the game is the management of the player's light. As mentioned above, this resource will constantly decrease at a slow rate,
and when completely exhausted, the player will lose the game. However, having too much light at a given time will also result in a game over,
meaning the player must carefully manage this resource. Rather than being displayed numerically, the amount of light the player has will be
graphically reflected in the player character. For example, the default state of the character could emit a light with bloom, which dims
or brightens based on how much light the player currently has.

As mentioned before, the only action available to the player is jumping. This will be done by pressing a key, which will most likely be the space bar.
Tapping and releasing the key immediately should result in a small jump, while holding the key for longer should result in a higher jump, to a degree.
Jumping may also consume light, with more powerful jumps requiring more. This adds another challenge for the player, in that they must manage their jumps
and light collection properly in order to avoid a game over.

The game should have two scenes: a main menu from which the player can start a new game or exit, and a play scene where the actual gameplay occurs.
The play scene should consist of the player character and the level, which is made up of various tiles on which the player moves, and collectibles
that grant the player light. The level consists of various tilesets that are randomly picked from a collection of premade templates. Gameplay should
theoretically be able to last indefinitely, so when the player nears the end of their current tileset, a new one is placed ahead of the player,
and any tilesets that are no longer visible should be removed from the level.

To avoid overwhelming the player, the difficulty of the game should gradually increase, based on how long the current run has lasted.
As an example, the first few tilesets may be quite simple, with mostly unbroken platforms and some obstacles, while later tilesets
might have considerably more obstacles or difficult jumps. The speed at which the tiles move will also gradually speed up.
The difficulty should be balanced around each run lasting about one to two minutes, on average.
Simple instructions will be displayed as part of the level at the start of each run, rather than having an explicit instruction popup.
This makes it so that the player does not learn the basics of the game through trial and error, but optimal strategies (if any exist)
are left for the player to discover themselves.

During gameplay, a stopwatch should also be displayed on the screen which tracks how long the current run has gone on for.
The final length of the run should be shown to the player when the run ends. This introduces a meta-goal where
players might be challenged to survive the longest.

\section{Aesthetics}

The visual theme of the game should be futuristic, resembling scenes that one might find in science fiction.
Proper lighting, as well as the use of post-processing effects like bloom, will be integral to producing this theme.

The graphics of the game will be predominantly created using 2D sprites. Certain visual effects, such as the indicator
for the player having too much light, may also use particle systems. Given the visual theme of the game, it is likely
that complex sprite animations will be unnecessary, and simple animations can be done using particle systems instead.

There will likely be some looping royalty-free background music added to the game,
which should also match the futuristic visual theme of the game.

Story is not really necessary for this type of game, so it will be left out.

\section{Player Experience}

The player experience for \textit{Lightrunner} should start out relatively relaxed, with it progressively becoming
more intense, suspenseful, and energetic. Players should be able to settle into a kind of rhythm, or flow state,
as the game progresses, where they are focused and in tune with the gameplay. The use of upbeat, electronic music
and futuristic neon graphics contribute to this energetic theme, and the game's music player was also coded so
that tracks fade in and out appropriately in order to avoid disrupting the player's flow.
The progression from a relaxed experience to a more intense and energetic one is also supported by the game's mechanics
becoming increasingly difficult as time goes on, with more complex tilesets being spawned that also increase in speed.

Players will likely feel frustrated when they lose. This is okay, as long as the player also feels that they
lost because of their own actions, rather than because the game made it outright impossible for them to continue.
The nature of using tilesets will help with this somewhat, as each tileset will have a recognizable pattern and appearance,
which will help the player decide on the best course of action based on their current situation. For example,
some tiles will force the player onto two different paths, with either a lot of light collectibles, or no light collectibles.
The player then has to decide which path to take based on how much light they currently have. There is also an aspect of
player skill as well, as the player must learn for themselves how far their jumps will take them, whether
they can land a jump properly based on the current speed, and so on.

The intended audience of the game could be anyone from kids to about middle-aged adults, as players will need to be able to
react relatively quickly to the game's mechanics. The audience will likely also lean more casual as opposed to ``hardcore'',
as the game is designed so it can be picked up rather quickly, and put back down again after a few runs. The fact that the
game tracks each individual run's duration but does not include any leaderboards will also mean that any competitive aspect
will likely be the player competing with themselves to increase their highest run duration, rather than competing against
other players directly.

\section{Changes from Original}

The biggest change from the original design is that the average length of a single run was decreased from five minutes to about one to two minutes.
The reason for this is that designing the tilesets to produce such long individual run times led to the gameplay being rather dull.
So now, the tilesets are designed so that an average run lasts from about one to two minutes, and there are enough tilesets that a player
should experience a decent variety in two or three runs. This means there is still about five minutes of content on average.

As for aesthetics, the only changes were that jumping no longer has a sound effect, and there is no story/lore either.
This is because they were viewed as unnecessary, and did not really fit in with the rest of the game either.

This design document was also updated to reflect the given feedback, as well as to document any third-party assets and other such things that were used.

\section{Works Used}

The use of bloom and post-processing was learned from a \href{https://www.youtube.com/watch?v=WiDVoj5VQ4c}{Brackeys tutorial on 2D glow}.

The main background shader with animated colour was heavily modified from \href{https://www.shadertoy.com/}{Shadertoy's default new shader},
and translated from GLSL to ShaderGraph.

\subsection{Third-Party Assets}

\subsubsection{Fonts}

\textbf{Comfortaa}
\begin{itemize}
    \item Licensed under OFL
    \item Sourced from \href{https://fonts.google.com/specimen/Comfortaa}{Google Fonts}
\end{itemize}

\textbf{Nova Square}
\begin{itemize}
    \item Licensed under OFL
    \item Sourced from \href{https://fonts.google.com/specimen/Nova+Square}{Google Fonts}
\end{itemize}

\subsubsection{Music}

"Furious Freak", "Super Power Cool Dude", "Voxel Revolution"\\
Kevin MacLeod (\href{https://incompetech.com}{incompetech.com})\\
Licensed under Creative Commons: By Attribution 3.\\
\url{http://creativecommons.org/licenses/by/3.0/}

\end{document}