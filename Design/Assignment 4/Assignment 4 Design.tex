\documentclass[10pt]{article}
\usepackage[utf8]{inputenc}
\usepackage[default]{lato}
\usepackage[T1]{fontenc}
\usepackage[margin=1in]{geometry}
\usepackage{listings}
\usepackage{graphicx}
\usepackage{hyperref}
\usepackage{xcolor}
\usepackage{amsmath}
\usepackage{amsthm}
\usepackage{amssymb}

\setlength{\parindent}{0em}
\setlength{\parskip}{1em}

\definecolor{linkcolour}{RGB}{0, 149, 255}
\hypersetup{
    colorlinks=true,
    linkcolor=linkcolour
}

\begin{document}

\title{\textit{Gloamwardens}}
\author{Luna Y.}
\date{
    SFWRENG 3GB3 -- Assignment 4\\[0.25cm]
    \today
}
\maketitle

\tableofcontents

\section{Basic Information}

Tower defence games have a number of common conventions: one prominent convention is that the player's units are static, unmoving towers
that are immune to damage from enemies. That being said, this is not always the case; in \textit{Orcs Must Die}, defences such as walls
or archers can be destroyed or killed by enemies, and if \textit{Kingdom Rush}, some towers will spawn friendly units that can be attacked by
enemies, or told by the player to gather at a designated location. However, even if some tower defence games feature RTS-like units in some
capacity, they are usually considered disposable or replaceable; losing them is usually a minor inconvenience, like in RTS games.

For this assignment, the game will be a subvert the tower defence convention mentioned above. Instead, the player's units will be
characters similar to those in tactical RPGs, rather than buildings, and the player will use these characters to autonomously defend
a number of converging paths, much like in tower defence games. The player is still limited in how much control they have over their units;
rather than being able to explicitly tell them to attack a specific enemy, they will instead give priorities, similar to how targeting
in \textit{Bloons TD 6} works. However, each unit will have progression and equipment that is somewhat similar to squad-based tactics games
like the newer \textit{XCOM} games, or role-playing games like \textit{Dungeons and Dragons}.

Specifically, the skill system takes inspiration from the skill tree of \textit{XCOM 2: War of the Chosen}'s faction units,
where the player may pick any of the potential skills for that unit's current rank, provided they have enough skill points.
However, there is one obvious difference: in \textit{Gloamwardens}, each skill costs exactly one skill point, to keep things simple.
Additionally, the class system is inspired by MMORPGs such as \textit{Final Fantasy XIV}, where each class fits into a particular role;
some are specialized for dealing damage, others are mainly healers, and others ``tanks'', who soak up damage from enemies in order
to protect other party members. The difference here, however, is that players are not strictly forced to have exactly two
damage dealers, one healer, and one tank; players may experiment with all kinds of party compositions, though it is
likely that the traditional MMO composition will still be the optimal one.

\section{Gameplay}

The game is meant to have multiple pre-made levels that are accessible at any time, with the first level also serving as a tutorial of sorts.
However, time constraints resulted in only one level making it to the assignment submission.
Gameplay in each level will follow the basic tower defence formula: there are one or more paths, waves of enemies appear at one end and travel to the other,
and if an enemy reaches the end, the player loses lives. However, instead of placing towers, the player will instead assemble parties of adventurers
and place them along the path. Adventurers in each party will autonomously fight enemies and support allies that are nearby. Players cannot directly
control adventurers, but they may adjust their targeting priorities in-between waves. The player is also able to signal parties to retreat, at which point
they are removed from the game area, and may be summoned again. The player may choose to immediately re-deploy the retreating party,
but since their health does not replenish, this may prove lethal to one or more adventurers.

At the start of a level, forms parties from adventurers that are selected from a pool.
Each adventurer has their own attributes such as base health, base mana, name, race, etc. Adventurers are
automatically assigned a class (and, by extension, a role of either tank, damage dealer, or support/healer), which determines what skills they can acquire,
what equipment they can use, and so on. If an adventurer dies from reaching zero health, they are permanently gone, which contrasts heavily with how
towers work in traditional tower defence games (since usually, they are either invincible or easily replaceable). As such, if an adventurer is in danger,
the player must weigh the benefits of having their party retreat, allowing enemies to pass to the player's next line of defence, or keeping them
in the fight, which can result in the adventurer or even their party dying. Adventurers earn experience by defeating enemies, and will level up
at various intervals, granting them skill points which they can use to unlock skills/perks that can either improve their attributes, or grant them
new attacks and abilities.

Players assemble parties of adventurers (most likely four, to emulate the standard party composition in MMORPGs, though instead of forcing two damage
dealers, a support, and a tank, they may have any amount of each role), and these parties (rather than individual adventurers) can then be placed
somewhere on the path during a level. All enemies that encounter the party will stop to fight the adventurers, and will continue to do so until the party
retreats, or either the enemies or the party members are defeated.

In each level, the player must defeat a certain number of enemy waves to complete it, at which point they may exit the level.
Because the player must spend time between waves to improve their adventurers, and possibly adjust party compositions, each wave
must be manually launched by the player.

Combat works as follows: characters pick a target nearby that satisfies their targeting conditions (e.g. if an adventurer is set to always target
the enemy with highest health, they will do so), and then, in real time, perform a set of attacks on them, with each one having its own cooldown timer.
Each character will have a basic attack, and adventurers can unlock more by using certain weapons or acquiring skills. Adventurers will generate
some kind of ``threat'' resource when attacking, and enemies will always attack whoever has the highest threat. This allows for more complex gameplay,
where players may have well-armoured adventurers that generate high amounts of threat to protect their more fragile party members. Each character
will have physical and ranged damage resistance values, which decrease the damage they take of the respective type, and may be increased
from skills or equipment.

\section{Aesthetics}

The game will be themed around a high fantasy setting. Because the only assets I could find were more ``dark'' and gloomy, rather than bright
and vivid, the visual style of the game will be darker as well. Any music used should be appropriate for a medieval high fantasy setting
(i.e. should be more instrumental/orchestral, rather than electronic), and ideally, the combat and ``build'' phases should have different music playing.
The former should have more energetic music, while the latter should have more relaxed or suspenseful music.

The premise of the game is that the player leads a company of adventurers whose goal is to safeguard the nighttime from monsters and other such villains
(hence the name \textit{Gloamwardens}). They do so by venturing into caves and other such places to prevent the monsters within from leaving
to attack the populace. This small bit of story should be conveyed to the player as part of the instructions, but there doesn't need to be
much of a narrative during gameplay. The adventurers, each of which has their own name, class, appearance, etc. also add an element of emergent
storytelling, which can be further supported by tracking various statistics. For example, the player might feel impacted if they lose one
of their adventurers, who had killed dozens or even hundreds of enemies, or who they've had since the beginning of a level and invested heavily in.

\section{Player Experience}

The player is rather detached from the combat of the game, so that will always be a kind of dampener on the intensity of the player experience.
Nevertheless, this game should take a lesson from \textit{Kingdom Rush} and aim to have an engaging combat phase where the player is not bored.
Even though the player is very limited in how they can interact with their adventurer parties during the combat phase, they should still feel invested
in how well their parties are faring. The music of the game can also help here: during combat, the music player will play more energetic tracks,
while out of combat, the music will be slower, and perhaps more mysterious.

Gameplay should also feel strategic. The player should be able to formulate a plan (e.g. what kind of adventurers to put in a party, where to place the party,
how should they advance their skills, etc.) and carry it out. There should be moments of suspense, especially during close fights, and moments of triumph,
possibly after said fights, or after defeating very powerful waves of enemies. Losing adventurers, especially ones the player was invested in,
should evoke a sense of loss. The need for planning is supported by the choice to not replenish the pool of available adventurers until \textit{after}
combat has concluded. This way, the player cannot just send party after party of adventurers at a wave of enemies, and must make do with the
parties they already have deployed (in addition to a limited number of possible reinforcements in the pool). Conveniently, this
also indirectly limits how quickly the player can grow their forces.

Another part of strategic gameplay is the idea that different classes of adventurers should synergize with each other. Normally, adventurers
will regenerate health very slowly over time by themselves, which encourages players to add healers to their parties to regenerate health faster.
These healers (as well as specialized damage-dealing classes) are also rather fragile, which in turn encourages the player to protect them
by using adventurers that can taunt and draw the attention of enemies.

\section{Changes from Original}

Unfortunately, some planned features did not make it into the implementation. The largest feature that was removed would be the ability
to acquire new equipment and re-equip adventurers. In the implementation, equipment still exists, but there is no way to gain new equipment
or replace what adventurers already have, due to time constraints.

Likewise, the ``endless mode'' with procedurally generated waves was also scrapped due to time constraints.

\section{Works Used}

% \subsection{Concepts}

\subsection{Code}

Like Assignment 2, the game features many script assets originally made for \textit{Nightscape},
which is still not released publicly. This includes camera functionality, object selection, as well as
random name generation. Many of the names themselves include segments generated from
\href{https://www.fantasynamegenerators.com/}{Fantasy Name Generators}.
The appropriate behaviours were also modified to use Unity's new input system, using concepts learned
from \href{https://www.youtube.com/watch?v=PsAbHoB85hM}{this tutorial} as well as the Unity documentation.

This game also builds on \textit{The Starlight Rebellion}'s version of the audio player originally made
for \textit{Lightrunner}.

The game also uses the \href{https://arongranberg.com/astar/}{A* Pathfinding Project} as its pathfinding algorithm,
which uses the \href{https://unity3d.com/legal/as_terms}{Standard Unity Asset Store EULA}. Movement, attacking,
and some other AI-related behaviour was heavily adapted from another one of my old personal projects called \textit{War of the Vale},
which is also not publicly available.

\subsection{Art}

\textbf{\textit{Dungeon Crawl Stone Soup}}
\begin{itemize}
    \item The game uses main and supplementary tilesets from \textit{Dungeon Crawl Stone Soup}, hosted \href{https://opengameart.org/content/dungeon-crawl-32x32-tiles}{here}
        and \href{https://opengameart.org/content/dungeon-crawl-32x32-tiles-supplemental}{here}, respectively.
    \item These assets are in the public domain, under the \href{https://creativecommons.org/publicdomain/zero/1.0/}{CC0 1.0} license.
\end{itemize}

\textbf{RPG Worlds Caves}
\begin{itemize}
    \item The game uses Szadi's RPG Worlds Caves tileset, hosted on the \href{https://assetstore.unity.com/packages/2d/environments/rpg-worlds-caves-167274}{Unity asset store}.
    \item These assets are licensed under the \href{https://unity3d.com/legal/as_terms}{Standard Unity Asset Store EULA}.
\end{itemize}

\textbf{Rogue Fantasy Castle}
\begin{itemize}
    \item The game uses Szadi's Rogue Fantasy Castle tileset, hosted on the \href{https://assetstore.unity.com/packages/2d/environments/rogue-fantasy-castle-164725}{Unity asset store}.
    \item These assets are licensed under the \href{https://unity3d.com/legal/as_terms}{Standard Unity Asset Store EULA}.
\end{itemize}

\subsection{Fonts}

\textbf{Alegreya SC}
\begin{itemize}
    \item Licensed under OFL
    \item Sourced from \href{https://fonts.google.com/specimen/Alegreya+SC}{Google Fonts}
\end{itemize}

\textbf{IM Fell Double Pica SC}
\begin{itemize}
    \item Licensed under OFL
    \item Sourced from \href{https://fonts.google.com/specimen/IM+Fell+Double+Pica+SC}{Google Fonts}
\end{itemize}

\subsection{Music}

Dreamy Piano Fantasy  by Rafael Krux\\
Link: \url{https://filmmusic.io/song/5635-dreamy-piano-fantasy-}\\
License: \url{https://filmmusic.io/standard-license}

Electric Cellos  by Rafael Krux\\
Link: \url{https://filmmusic.io/song/5636-electric-cellos-}\\
License: \url{https://filmmusic.io/standard-license}

Land Of Magic  by Rafael Krux\\
Link: \url{https://filmmusic.io/song/5679-land-of-magic-}\\
License: \url{https://filmmusic.io/standard-license}

Midnight Magic  by Rafael Krux\\
Link: \url{https://filmmusic.io/song/5426-midnight-magic-}\\
License: \url{https://filmmusic.io/standard-license}

% Dungeon Crawl Stone Soup assets, Szadi RPG Cave and Rogue Castle.

\end{document}