\documentclass[10pt]{article}
\usepackage[utf8]{inputenc}
\usepackage[default]{lato}
\usepackage[T1]{fontenc}
\usepackage[margin=1in]{geometry}
\usepackage{listings}
\usepackage{graphicx}
\usepackage{hyperref}
\usepackage{xcolor}
\usepackage{amsmath}
\usepackage{amsthm}
\usepackage{amssymb}

\setlength{\parindent}{0em}
\setlength{\parskip}{1em}

\definecolor{linkcolour}{RGB}{0, 149, 255}
\hypersetup{
    colorlinks=true,
    linkcolor=linkcolour
}

\begin{document}

\title{\textit{Lightrunner}}
\author{Luna Y.}
\date{
    SFWRENG 3GB3 -- Assignment 1\\[0.25cm]
    \today
}
\maketitle

\tableofcontents

\section{Basic Information}

In this game, the player will play as an amorphous entity composed of light. The player character is constantly travelling forwards,
and the only action available to the player is jumping. The player character is constantly shrinking due to expending their light,
and must collect more by moving close enough in order to survive and continue playing. However, collecting too much light will
cause the player to become unstable, and they will eventually dissipate once a certain point is reached, causing a game over as well.

The goal of the game is to survive as long as possible, which will require the player to collect enough light to survive,
while taking care to avoid obstacles and not collecting too much light.

In terms of level design, the game world will resemble a standard 2D platformer, except the player is constantly moving forwards (i.e. to the right),
and the only way to travel vertically is by jumping. The tiles of the level, as well as the light being collected by the player,
will be procedurally generated according to a collection of premade tilesets; this should allow for more variety while also keeping
the level structured and less chaotic.

\end{document}